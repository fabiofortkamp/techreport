\documentclass[article]{techreport}

% ---
% INITIAL CONFIGURATION
% These packages must later go into a single package
\usepackage[output-decimal-marker={,},per-mode=symbol]{siunitx}
\DeclareSIUnit{\mol}{mol}
\DeclareSIUnit{\molecule}{molécula}
\DeclareSIUnit{\molecules}{moléculas}
\DeclareSIUnit{\poise}{P}

\usepackage{esdiff}
\newcommand{\diffd}[1]{\ensuremath{\mathrm{d}#1}}

% some common commands that I use in technical reports
\newcommand{\mrate}[1]{\ensuremath{\dot{m}_{#1}}}

\newcommand{\overbar}[1]{\mkern 1.5mu\overline{\mkern-1.5mu#1\mkern-1.5mu}\mkern 1.5mu}

\newcommand{\average}[1]{\overline{#1}}

\newcommand{\solubility}{\ensuremath{\omega_R}}

\newcommand{\solubilitym}{\ensuremath{\tilde{\omega}_R}}

% ---
% FILL THIS INFORMATION

\titulo{A classe \textsf{techreport}}
\autor{Fábio Pinto Fortkamp}
\local{Florianópolis}
\data{Janeiro de 2015}

\instituicao{}

% --- 
% COVER TEMPLATE
\renewcommand{\imprimircapa}{%
  \begin{capa}%
    
    \center
    {\ABNTEXchapterfont\Large\imprimirinstituicao}
    
    \vspace*{\fill}
    {\ABNTEXchapterfont\bfseries\LARGE\imprimirtitulo}
    \par
    \par
    {\ABNTEXchapterfont\large Versão de artigo}
    \vspace*{\fill}

    \vspace*{\fill}
    {\ABNTEXchapterfont\large\imprimirautor}
    \vspace*{\fill}

    {\ABNTEXchapterfont\large\imprimirlocal}
    \par
    {\ABNTEXchapterfont\large\imprimirdata}

    \vspace*{1cm}

  \end{capa}
}

% ---
% PDF SETUP

\configurepdf

% ---
% ADITIONAL COMMANDS
\newcommand{\trep}{\textsf{techreport}}

\begin{document}

\pretextual
% ---
% PRINT THE COVER?
\imprimircapa
%\maketitle

% this puts the table of contetns in the PDF bookmarks
%
% with just the \tableofcontents macro, the bookmark is still produced, but it directs the PDF
% to *below* the table of contents name
%
% the starred version then excludes the bookmark, and we manually add a new one, that will redirect the reader
% to table of contents title (like "Contents")
\pdfbookmark[0]{\contentsname}{toc}
\tableofcontents*
% \vspace*{\baselineskip}
% \pdfbookmark[0]{\listfigurename}{lof}
% \listoffigures*

% ---
% BEGIN DOCUMENT
\textual


\section{Introdução}
\label{sec:introducao}

Este document descreve a classe \trep, uma classe baseada na classe \textsf{abntex2} \cite{bib:abntex2classe} , já com algumas configurações padronizadas. Na linguagem de hoje, esta é uma classe ``opinionada''.

As principais características de \trep{} são:

\begin{enumerate}
\item Carregar a classe \textsf{abntex2} com algumas opções como default, e carrgar o pacote \textsf{abntex2cite} \cite{abntex2cite}.
\item Carregar os principais pacotes usados na maioria dos documentos técnicos em português, como \textsf{cmap}, \textsf{graphicx}, \textsf{inputenc}, \textsf{fontenc}
\item Configurar uma geometria de página, com margens e espaçamento entre linhas
\item Configurar novas fontes
\end{enumerate}

O principal objetivo desta classe é que o usuário seja capaz de configurar um documento técnico da maneira mais rápida possível, seguindo as orientações da ABNT, e tendo de configurar apenas pacotes específicos ao documento sendo criado, deixando os aspectos gerais (como a geometria, citada anteriormente) para a classe. Ao mesmo tempo, o uso de fontes alternativas confere um toque pessoal ao documento, sem perder a seriedade.

Possíveis usos pensados para a classe são:

\begin{itemize}
\item Trabalhos de disciplinas de ensino superior
\item Documentação para algum produto/programa/procedimento
\item Um artigo científico comum
\item Relatório técnico (de onde deriva o nome)
\end{itemize}

Para ser claro: \trep{} é apenas uma personalização de \textsf{abntex2}; todo o trabalho pesado é feito por esta classe, ao qual os estudantes e profissionais brasileiros que precisam lidar com normas técnicas agradecem.

\section{Configurações}
\label{sec:configuracoes}

\textbf{Versão atual de techreport: v0.7}.

Nesta versão, existem algumas pequenas configurações manuais que precisam ser feitos pelo usuário. Um template mínimo para a classe poderia ser:

\begin{verbatim}
\documentclass[article]{techreport}

% ---
% FILL THIS INFORMATION

\titulo{}
\autor{}
\local{
\data{}

\instituicao{}

% --- 
% COVER TEMPLATE
\renewcommand{\imprimircapa}{%
  \begin{capa}%
    \center
    {\ABNTEXchapterfont\Large\imprimirinstituicao}
    
    \vspace*{\fill}
    {\ABNTEXchapterfont\bfseries\LARGE\imprimirtitulo}
    \vspace*{\fill}

    \vspace*{\fill}
    {\ABNTEXchapterfont\large\imprimirautor}
    \vspace*{\fill}

    {\ABNTEXchapterfont\large\imprimirlocal}
    \par
    {\ABNTEXchapterfont\large\imprimirdata}

    \vspace*{1cm}

  \end{capa}
}

% ---
% PDF SETUP

\configurepdf


% ---
% BEGIN DOCUMENT


\begin{document}

\pretextual
% ---
% PRINT THE COVER?
\imprimircapa

\pdfbookmark[0]{\contentsname}{toc}
\tableofcontents*

\textual

% CONTENTS GO HERE

\postextual
\bibliography{}

\end{document}

\end{verbatim}

O usuário pode usar algum recurso de templates; alguns editores têm recursos de \emph{snippets}; usuários de OS X podem usar um programa como o TextExpander.

As principais configurações são referentes aos metadados do document, como o título, autor, local etc. Este template inclui uma possível versão para uma capa simples. É importante ressaltar que \trep{} se baseia num modelo de artigo, na verdade, sendo estruturado em seções e não em capítulos. Versões futuras poderão permitir uma versão mais compatível com documentos complexos, incluindo elementos pré-textuais. 

Caso o usuário não deseja uma capa, pode comentar o commando \verb=\imprimircapa= e incluir o comando \verb=\maketitle=.

O comando \verb=\configurepdf= configura o arquivo PDF final para conter os metadados (título, autor etc) do documento.

Por último, é importante observar os comandos \verb=\pretextual=, \verb=\textual= e \verb=\postextual=, que indicam o início do documento e de elementos pré-textuais (como a capa) e pós-textuais, como referências.

\section{Personalizações feitas}
\label{sec:pers-feit}

\subsection{Geometria}
\label{sec:geometria}

\begin{itemize}
\item O tamanho padrão é A4
\item Tamanho de fonte de \SI{2}{pt}
\item \trep{} configura margens superior e esquerda de \SI{3}{\centi\meter} e inferior e direita de \SI{2}{\centi\meter}, seguindo a norma ABNT NBR 14724:2011 \cite{NBR14724:2011}.
\item O espaçamento entrelinhas é de \num{1,5}, segundo a mesma norma.
\item Não há espaço entre parágrafos.
\item A tabulação é de \SI{0,75}{\centi\meter}.
\item O comando \verb=\textual= define um estilo de página com o número de página no canto superior direito, como pede a norma
\item É usado um sumário customizado, com os diferentes níveis com diferentes níveis de espaçamento, diferentemente do que pede a norma mas infinitamente mais bonito
\end{itemize}

\subsection{Versão de artigo}
\label{sec:article}

Para produzir uma versão ``de artigo'' da classe, passe a opção \verb=article= para a classe. Isso tem alguns efeitos, principalmente:

\begin{itemize}
\item As seções são a principal divisão do documento
\item O cabeçalho da página não exibe nenhum texto, apenas o número da página
\item O sumário é ajustado para levar em conta que as seções são a principal unidade; inclusive, as referências bibliográficas são impressas no sumário como uma seção
\end{itemize}

Este template é produzido desta forma.

Para documentos produzidos dessa forma, geralmente não há muito sentido em criar lista de figuras, tabelas etc, mas é possível. Apenas é preciso tomar cuidado com espaçamentos antes e depois dessa lista e com os bookmarks do PDF.

\subsection{Citações}
\label{sec:citacoes}

É utilizado a configuração de citação do tipo (NOME, ANO) do \textsf{abntex2-cite}. Como configuração adicional, trabalhos com muitos autores não são abreviados como ``et al'' na lista de referências.

\subsection{Fontes}
\label{sec:fontes}

A fonte principal do texto é a Palatino, com o pacote \textsf{pxfonts}

Como referência, uma equação em linha fica $a + \sqrt{b^2} = \exp \gamma$, e uma equação em parágrafo fica:

\begin{equation}
  \label{eq:1}
  \left( 1 + \frac{r}{R} \right) \sin \theta_i = 2 \beta \nabla \cdot \vec{V} 
\end{equation}

\subsubsection{Fontes de seções}
\label{sec:fontes-de-secoes}

Esta subseção é apenas para ver como os títulos de seções são produzidos até o nível de subsubseção.

\subsection{Pacotes carregados}
\label{sec:pacotes-carregados}

Além dos considerados essenciais pelo \textsf{abntex2}:
\begin{itemize}
\item \textsf{amsmath}
\end{itemize}


\subsection{English}
\label{sec:english}

Para escrever o texto em inglês, passe as opções \texttt{[brazil,english]} à classe. Isso tem o efeito de mudar as legendas e referências a figuras, tabelas e equações, como pode ser visto na \autoref{fig:t} (compile este documento com estas opções para ver o resultado). Veja também uma referência à \autoref{eq:1} e à \autoref{sec:pacotes-carregados}.

O título das referências bibliográficas também é alterado, e é usado o ``\&'' nas citações com mais de um autores (como em \citeonline{abntex2cite}). 

\begin{figure}[!ht]
  \centering
  
  \caption{Test}
  \label{fig:t}
\end{figure}

\postextual

\bibliography{techreport-ref}

\end{document}

%%% Local Variables: 
%%% mode: latex
%%% TeX-master: t
%%% End: 
